%!TEX program = xelatex

\documentclass{progartcn}
\usepackage{graphicx}
\usepackage[dvipsnames]{xcolor}
\usepackage{wrapfig}
\usepackage{enumerate}
\usepackage{amsmath,mathrsfs,amsfonts}
\usepackage{booktabs}
\usepackage{tabularx}
\usepackage{colortbl}
\usepackage{multirow,makecell}
\usepackage{multicol}
\usepackage{ulem} % \uline
\usepackage{listings}
\usepackage{tikz}
\usepackage{tcolorbox}
\usepackage{fontawesome}


\title{\bfseries\sffamily
  FSK调制与解调\\
  \normalfont\zihao{-3}
   通信原理实验
}
\author{[无52]\ \textbf{王冠儒}\quad \&\quad [无52]\ \textbf{万子牛} \\ \faGithubAlt~ https://github.com/WisdomFusion}
\date{}


\begin{document}

\sloppy % 解决中英文混排文字超出边界问题


\maketitle
\thispagestyle{empty}


\section{实验目的}
\label{newfeatures}

\begin{enumerate}
  \item 了解 FSK 调制与解调原理;

  \item 熟悉 FPGA 实验平台;

  \item 熟悉编码与解码;

  \item 利用 Quartus II 软件、Altera 实验平台实现 FSK 调制与解调的程序设计。
\end{enumerate}

\section{实验原理}

本次实验中,我们利用实验板实现了一个基于 FSK 方式的通信系统。具体操作步骤如下:

\begin{enumerate}
  \item \textbf{编码}。用户从实验板上的按键输入一组 4 位原码,这 4 位原码经过 FPGA 编码后,形成 10 位码元,即一帧编码结果;在这 10 位中,前 3 位固定为 110,在解码时用来识别一帧的开头;中间的 4 位是原码,最后 3 位是汉明码。
  
  \item \textbf{调制}。这 10 位在编码后,串行输入到调制模块。调制模块的调制方式是 FSK 调制。

  \item \textbf{解调}。调制完成后,输出到数模转换模块;数模转换与模数转换模块相连,然后输出到解调模块。

  \item \textbf{解码}。经过调制的编码结果经过解调后,输出到解码部分。被解码后,输出结果通过 4 个指示灯来显示,灯暗代表 0 ,灯亮代表 1 。同时输出到指示灯的还有一位示错信号,用来指示是否接收到的信号是否有错。

\end{enumerate}

\section{实验内容与仿真}

\subsection{调制模块(王冠儒实现)}

调制模块本质上是一个多路选择器。它首先通过原始时钟信号分频生成两个时钟信号,分别是1分频和2分频。而多路选择器的控制信号则是编码结果信号。当编码信号为0时,输出1分频时钟;当编码信号为1时,输出2分频时钟。
其系统原理框图如下:



该部分代码如下:

\begin{lstlisting}[language=TypeScript,caption={FSK.v}]
module FSK (FSK_in, FSK_out, clk, rst_n);
  input FSK_in;
  input clk;
  input rst_n;
  output FSK_out;

  reg [1:0] cnt;
  reg clk_0, clk_1;

  always @(posedge clk or negedge rst_n)
  begin
    if(~rst_n)
      cnt <= 2'd0;
    else
      cnt <= (cnt < 2'd3)? (cnt + 2'd1): 2'd0;
  end

  always @(cnt)
  begin
    case(cnt)
      2'd0: begin clk_0 = 1; clk_1 = 1; end 
      2'd1: begin clk_0 = 0; clk_1 = 1; end
      2'd2: begin clk_0 = 1; clk_1 = 0; end
      2'd3: begin clk_0 = 0; clk_1 = 0; end
    endcase
  end

  assign FSK_out = (~FSK_in)? clk_0: clk_1;
  

endmodule
\end{lstlisting}

\subsection{解调模块(万子牛实现)}

解调模块本质上是一个计数器。由于原始时钟的一分频代表信号0,二分频代表信号1;而编码信号则是由原始时钟的十六分频,即一个码字的持续时长为原始信号的16个周期。因此,只需在原始信号的16个周期内进行计数,每16个周期结束后对计数器进行检查:
\begin{enumerate}
  \item 若计数器结果为\textbf{16},则代表该码字为0;
  \item 若计数器结果为\textbf{8},则代表该码字为1;
\end{enumerate}

为了使模块对噪声具有更强的鲁棒性,对以上策略进行修改:
\begin{enumerate}
  \item 若计数器结果大于\textbf{12},则代表该码字为0;
  \item 若计数器结果小于\textbf{12},则代表该码字为1;
\end{enumerate}

其系统原理框图如下:


该部分代码如下:

\begin{lstlisting}[language=TypeScript,caption={deFSK.v}]
module deFSK (deFSK_in, deFSK_out, clk, rst_n);
  input deFSK_in;
  input clk;
  input rst_n;
  output deFSK_out;

  reg [4:0] cnt_period;
  reg [4:0] c_cnt_in, n_cnt_in;
  reg c_data, p_data;
  reg c_out, n_out;

  always @(posedge clk or negedge rst_n)
  begin
    if(~rst_n)
    begin
      cnt_period <= 5'd0;
      c_cnt_in <= 5'd0;
      c_data <= 1'd0;
      p_data <= 1'd0;
      c_out <= 1'd0;
    end
    else
    begin
      cnt_period <= (cnt_period < 5'b11111)? (cnt_period + 5'd1): 5'd0;
      c_cnt_in <= n_cnt_in;
      c_data <= deFSK_in;
      p_data <= c_data;
      c_out <= n_out;
    end
  end

  always @(c_cnt_in, c_data, p_data)
  begin
    if(p_data == 1'b0 && c_data == 1'b1)
      n_cnt_in = (cnt_period == 5'd0)? 5'd0: (c_cnt_in + 5'd1);
    else
      n_cnt_in = (cnt_period == 5'd0)? 5'd0: c_cnt_in;
  end

  always @(c_cnt_in, cnt_period, c_out)
  begin
    if(cnt_period == 5'd0)
      n_out = (c_cnt_in <= 5'd12)? 1'b1: 1'b0;
    else
      n_out = c_out;
  end

  assign deFSK_out = c_out;

endmodule
\end{lstlisting}

\subsection{顶层封装(万子牛实现)}

系统框图如下:

实现代码如下:

\begin{lstlisting}[language=TypeScript,caption={FSKSystem.v}]
module FSKSystem(inputData, outputData, clk, wr, rst_n, encode_en);

    input clk, rst_n, encode_en;
    input [3:0] inputData;
    output wr;
    output [3:0] outputData;

    wire clkCode;
    wire dataToFSK;
    wire dataToDeFSK;
    wire dataToDecoder;

    clk_code codeClk(clk, clkCode, rst_n);
    encode encoder(inputData, dataToFSK, encode_en, clkCode, rst_n);
    FSK FSKer(dataToFSK, dataToDeFSK, clk, rst_n);
    deFSK deFSKer(dataToDeFSK, dataToDecoder, clk, rst_n);
    decode decoder(dataToDecoder, outputData, wr, clkCode, rst_n);

endmodule
\end{lstlisting}

\subsection{Testbench仿真(万子牛实现)}

\subsubsection{编码器模块}
\begin{lstlisting}[language=TypeScript,caption={encode\_tb.v}]
module encode_tb;

  reg clk;
  reg[3:0] inputData;
  reg encode_en;
  reg rst_n;
  wire outputData;

  encode encoder(inputData, outputData, encode_en, clk, rst_n);

  always
    begin
      #5 clk <= ~clk;
    end

  initial
    begin
      clk <= 1'b0;
      inputData <= 4'b0101;
      encode_en <= 1'b0;
      rst_n <= 1'b0;
      #100
      encode_en <= 1'b1;
      rst_n <= 1'b1;
    end

endmodule
\end{lstlisting}

\subsubsection{分频器模块}
\begin{lstlisting}[language=TypeScript,caption={clk\_tb.v}]
module clk_tb;

    reg clk, rst_n;
    wire clk_output;

    clk_code codeClk(clk, clk_output, rst_n);

    always
        begin
            #5 clk <= ~clk;
        end

    initial
    begin
      clk <= 1'b0;
      rst_n <= 1'b0;

      #40 rst_n <= 1'b1;

    end
        
endmodule
\end{lstlisting}

\subsubsection{调制模块}
\begin{lstlisting}[language=TypeScript,caption={FSK\_tb.v}]
module FSK_tb;
  
  reg inputData, clk, rst_n;
  wire outputData;

  FSK FSKer(inputData, outputData, clk, rst_n);

  always
    begin
      #5 clk <= ~clk;
    end

  initial
    begin
      clk <= 1'b0;
      rst_n <= 1'b0;
      inputData <= 1'b0;

      #320 inputData <= 1'b1;
      #320 inputData <= 1'b0;
      #320 inputData <= 1'b1;
      #320 inputData <= 1'b1;
      #320 inputData <= 1'b0;
      #320 inputData <= 1'b0;
      #320 inputData <= 1'b0;
      #320 inputData <= 1'b1;
      #320 inputData <= 1'b1;
      #320 inputData <= 1'b1;

      #320 rst_n <= 1'b1;

      #320 inputData <= 1'b1;
      #320 inputData <= 1'b0;
      #320 inputData <= 1'b1;
      #320 inputData <= 1'b1;
      #320 inputData <= 1'b0;
      #320 inputData <= 1'b0;
      #320 inputData <= 1'b0;
      #320 inputData <= 1'b1;
      #320 inputData <= 1'b1;
      #320 inputData <= 1'b1;
    end
endmodule
\end{lstlisting}

\subsubsection{解调模块}
\begin{lstlisting}[language=TypeScript,caption={deFSK\_tb.v}]
module deFSK_tb;

    reg inputSignal, clk, rst_n;
    wire outputData;

    deFSK deFSKer(inputSignal, outputData, clk, rst_n);

    always
        begin
            #5 clk <= ~clk;
        end
    
    initial
        begin
            clk <= 1'b0;
            inputSignal <= 1'b0;
            rst_n <= 1'b0;

            //rst_n test
            #20 inputSignal <= 1'b1;
            #20 inputSignal <= 1'b0;
            #20 inputSignal <= 1'b1;

            rst_n <= 1'b1;
            //code = 1
            #20 inputSignal <= 1'b1;
            #20 inputSignal <= 1'b0;
            #20 inputSignal <= 1'b1;
            #20 inputSignal <= 1'b0;
            #20 inputSignal <= 1'b1;
            #20 inputSignal <= 1'b0;
            #20 inputSignal <= 1'b1;
            #20 inputSignal <= 1'b0;
            #20 inputSignal <= 1'b1;
            #20 inputSignal <= 1'b0;
            #20 inputSignal <= 1'b1;
            #20 inputSignal <= 1'b0;
            #20 inputSignal <= 1'b1;
            #20 inputSignal <= 1'b0;
            #20 inputSignal <= 1'b1;
            #20 inputSignal <= 1'b0;

            //code = 0
            #10 inputSignal <= 1'b1;
            #10 inputSignal <= 1'b0;
            #10 inputSignal <= 1'b1;
            #10 inputSignal <= 1'b0;
            #10 inputSignal <= 1'b1;
            #10 inputSignal <= 1'b0;
            #10 inputSignal <= 1'b1;
            #10 inputSignal <= 1'b0;
            #10 inputSignal <= 1'b1;
            #10 inputSignal <= 1'b0;
            #10 inputSignal <= 1'b1;
            #10 inputSignal <= 1'b0;
            #10 inputSignal <= 1'b1;
            #10 inputSignal <= 1'b0;
            #10 inputSignal <= 1'b1;
            #10 inputSignal <= 1'b0;
            #10 inputSignal <= 1'b1;
            #10 inputSignal <= 1'b0;
            #10 inputSignal <= 1'b1;
            #10 inputSignal <= 1'b0;
            #10 inputSignal <= 1'b1;
            #10 inputSignal <= 1'b0;
            #10 inputSignal <= 1'b1;
            #10 inputSignal <= 1'b0;
            #10 inputSignal <= 1'b1;
            #10 inputSignal <= 1'b0;
            #10 inputSignal <= 1'b1;
            #10 inputSignal <= 1'b0;
            #10 inputSignal <= 1'b1;
            #10 inputSignal <= 1'b0;
            #10 inputSignal <= 1'b1;
            #10 inputSignal <= 1'b0;

            //code = 1
            #20 inputSignal <= 1'b1;
            #20 inputSignal <= 1'b0;
            #20 inputSignal <= 1'b1;
            #20 inputSignal <= 1'b0;
            #20 inputSignal <= 1'b1;
            #20 inputSignal <= 1'b0;
            #20 inputSignal <= 1'b1;
            #20 inputSignal <= 1'b0;
            #20 inputSignal <= 1'b1;
            #20 inputSignal <= 1'b0;
            #20 inputSignal <= 1'b1;
            #20 inputSignal <= 1'b0;
            #20 inputSignal <= 1'b1;
            #20 inputSignal <= 1'b0;
            #20 inputSignal <= 1'b1;
            #20 inputSignal <= 1'b0;


            //code = 0 (add noise)
            #10 inputSignal <= 1'b1;
            #10 inputSignal <= 1'b0;
            #10 inputSignal <= 1'b1;
            #10 inputSignal <= 1'b0;
            #10 inputSignal <= 1'b1;
            #10 inputSignal <= 1'b0;
            #3  inputSignal <= 1'b1;
            #4  inputSignal <= 1'b0;
            #3  inputSignal <= 1'b1;
            #10 inputSignal <= 1'b0;
            #10 inputSignal <= 1'b1;
            #10 inputSignal <= 1'b0;
            #10 inputSignal <= 1'b1;
            #10 inputSignal <= 1'b0;
            #10 inputSignal <= 1'b1;
            #10 inputSignal <= 1'b0;
            #10 inputSignal <= 1'b1;
            #10 inputSignal <= 1'b0;
            #10 inputSignal <= 1'b1;
            #10 inputSignal <= 1'b0;
            #10 inputSignal <= 1'b1;
            #10 inputSignal <= 1'b0;
            #10 inputSignal <= 1'b1;
            #10 inputSignal <= 1'b0;
            #3  inputSignal <= 1'b1;
            #4  inputSignal <= 1'b0;
            #3  inputSignal <= 1'b1;
            #10 inputSignal <= 1'b0;
            #10 inputSignal <= 1'b1;
            #10 inputSignal <= 1'b0;
            #10 inputSignal <= 1'b1;
            #10 inputSignal <= 1'b0;
            #10 inputSignal <= 1'b1;
            #10 inputSignal <= 1'b0;
            #10 inputSignal <= 1'b1;
            #10 inputSignal <= 1'b0;

            //code = 1 (add noise)
            #20 inputSignal <= 1'b1;
            #20 inputSignal <= 1'b0;
            #20 inputSignal <= 1'b1;
            #20 inputSignal <= 1'b0;
            #5  inputSignal <= 1'b1;
            #3  inputSignal <= 1'b0;
            #12 inputSignal <= 1'b1;
            #20 inputSignal <= 1'b0;
            #20 inputSignal <= 1'b1;
            #20 inputSignal <= 1'b0;
            #20 inputSignal <= 1'b1;
            #20 inputSignal <= 1'b0;
            #20 inputSignal <= 1'b1;
            #20 inputSignal <= 1'b0;
            #5  inputSignal <= 1'b1;
            #3  inputSignal <= 1'b0;
            #12 inputSignal <= 1'b1;
            #20 inputSignal <= 1'b0;
            #20 inputSignal <= 1'b1;
            #20 inputSignal <= 1'b0;


        end
endmodule
\end{lstlisting}

\subsubsection{解码器模块}
\begin{lstlisting}[language=TypeScript,caption={decode\_tb.v}]
module decode_tb;
  
  reg inputData, clk;
  reg rst_n;
  wire outputData, wrg_show;

  decode decoder(inputData, outputData, wrg_show, clk, rst_n);

  always
    begin
      #20 clk <= ~clk;
    end

  initial
    begin
      clk <= 1'b0;
      inputData <= 1'b0;
      rst_n <= 1'b0;

      #40 inputData <= 1'b1;
      #40 inputData <= 1'b1;
      #40 inputData <= 1'b0;
      #40 inputData <= 1'b1;
      #40 inputData <= 1'b0;
      #40 inputData <= 1'b0;
      #40 inputData <= 1'b0;
      #40 inputData <= 1'b1;
      #40 inputData <= 1'b1;
      #40 inputData <= 1'b1;

      //enable
      #40 rst_n <= 1'b1;
      #40 inputData <= 1'b0;

      //code = 1000|111 (Correct)
      #40 inputData <= 1'b1;
      #40 inputData <= 1'b1;
      #40 inputData <= 1'b0;
      #40 inputData <= 1'b1;
      #40 inputData <= 1'b0;
      #40 inputData <= 1'b0;
      #40 inputData <= 1'b0;
      #40 inputData <= 1'b1;
      #40 inputData <= 1'b1;
      #40 inputData <= 1'b1;

      #40 inputData <= 1'b0;

      //code = 1001|111 (Can be corrected)
      #40 inputData <= 1'b1;
      #40 inputData <= 1'b1;
      #40 inputData <= 1'b0;
      #40 inputData <= 1'b1;
      #40 inputData <= 1'b0;
      #40 inputData <= 1'b0;
      #40 inputData <= 1'b1;
      #40 inputData <= 1'b1;
      #40 inputData <= 1'b1;
      #40 inputData <= 1'b1;

      #40 inputData <= 1'b0;

      //code = 1010|111 (Can be corrected)
      #40 inputData <= 1'b1;
      #40 inputData <= 1'b1;
      #40 inputData <= 1'b0;
      #40 inputData <= 1'b1;
      #40 inputData <= 1'b0;
      #40 inputData <= 1'b1;
      #40 inputData <= 1'b0;
      #40 inputData <= 1'b1;
      #40 inputData <= 1'b1;
      #40 inputData <= 1'b1;

      #40 inputData <= 1'b0;

      //code = 0010|111 (Wrong)
      #40 inputData <= 1'b1;
      #40 inputData <= 1'b1;
      #40 inputData <= 1'b0;
      #40 inputData <= 1'b0;
      #40 inputData <= 1'b0;
      #40 inputData <= 1'b1;
      #40 inputData <= 1'b0;
      #40 inputData <= 1'b1;
      #40 inputData <= 1'b1;
      #40 inputData <= 1'b1;

      #40 inputData <= 1'b0;
      
      //code = 1100|111 (Can be corrected)
      #40 inputData <= 1'b1;
      #40 inputData <= 1'b1;
      #40 inputData <= 1'b0;
      #40 inputData <= 1'b1;
      #40 inputData <= 1'b1;
      #40 inputData <= 1'b0;
      #40 inputData <= 1'b0;
      #40 inputData <= 1'b1;
      #40 inputData <= 1'b1;
      #40 inputData <= 1'b1;

      #40 inputData <= 1'b0;

      //code = 1011|111 (Wrong, Would be corrected to 1111)
      #40 inputData <= 1'b1;
      #40 inputData <= 1'b1;
      #40 inputData <= 1'b0;
      #40 inputData <= 1'b1;
      #40 inputData <= 1'b0;
      #40 inputData <= 1'b1;
      #40 inputData <= 1'b1;
      #40 inputData <= 1'b1;
      #40 inputData <= 1'b1;
      #40 inputData <= 1'b1;

      #40 inputData <= 1'b0;

      //code = 0000|111 (Can be corrected)
      #40 inputData <= 1'b1;
      #40 inputData <= 1'b1;
      #40 inputData <= 1'b0;
      #40 inputData <= 1'b0;
      #40 inputData <= 1'b0;
      #40 inputData <= 1'b0;
      #40 inputData <= 1'b0;
      #40 inputData <= 1'b1;
      #40 inputData <= 1'b1;
      #40 inputData <= 1'b1;
    end
endmodule
\end{lstlisting}

\subsubsection{顶层封装}
\begin{lstlisting}[language=TypeScript,caption={FSKSystem\_tb.v}]
module FSKSystem_tb;

    reg clk, rst_n, encoder_en;
    reg [3:0] inputData;
    wire wr;
    wire [3:0] outputData;

    FSKSystem FSKSystemTest(inputData, outputData, clk, wr, rst_n, encoder_en);

    always
        begin
          #5 clk <= ~clk;
        end

    initial begin
        clk <= 1'b0;
        inputData <= 4'b1001; 
        rst_n <= 1'b0;
        encoder_en <= 1'b0;

        #20 rst_n <= 1'b1;
        #20 encoder_en <= 1'b1;
    end
    
endmodule
\end{lstlisting}

可以看到,仿真结果与预期符合,设计的模块没有问题。

\section{FSK系统加噪测试}

由于verilog无法实现产生随机信号,因此无法产生随机噪声。故我们加噪声的方式是将编码结果的某一位固定取反,然后观察解码信号是否正确。代码只需对\textbf{encode.v}略作修改即可:

\begin{lstlisting}[language=TypeScript,caption={encode.v}]
module encode (data_in, data_out, encode_en, clk, rst_n);
  input [3:0] data_in;
  input clk;
  input rst_n;
  input encode_en;
  output data_out;

  reg [3:0] c_state, n_state;
  reg [3:0] c_data, n_data;
  reg data_out;

  always @(posedge clk or negedge rst_n)
  begin
    if(~rst_n)
    begin
      c_state <= 4'd0;
      c_data <= 4'd0;
    end
    else
    begin
      c_state <= n_state;
      c_data <= n_data;
    end
  end

  always @(c_state, encode_en, c_data)
  begin
    if(c_state == 4'd0)
    begin
      n_state = (encode_en)? 4'd1:4'd0;
      n_data = data_in;
    end
    else if(c_state < 4'd10)
    begin
      n_state = c_state + 4'd1;
      n_data = c_data;
    end
    else
    begin
      n_state = 4'd0;
      n_data = 4'd0;
    end
  end

  always @(c_state, c_data)
  begin
    case(c_state)
      4'd0: data_out = 1'b0;
      4'd1: data_out = 1'b1;
      4'd2: data_out = 1'b1;
      4'd3: data_out = 1'b0;
      4'd4: data_out = c_data[3];
      4'd5: data_out = c_data[2];
      4'd6: data_out = ~c_data[1];
      4'd7: data_out = c_data[0];
      4'd8: data_out = c_data[3] ^ c_data[2] ^ c_data[1];
      4'd9: data_out = c_data[3] ^ c_data[2] ^ c_data[0];
      4'd10: data_out = c_data[3] ^ c_data[1] ^ c_data[0];
      default: data_out = 1'b0;
    endcase
  end

endmodule
\end{lstlisting}


测试结果如下图:

可以看到,虽然编码结果的一位被固定取反,解码信号仍然得到了正确的编码结果。


\section{实验总结与分工}

本次实验我们两人共同完成了这个FSK调制与解调的编码系统,虽然组里的人数比一般其他小组要少一人,但是凭借着成员扎实的verilog基础,我们仍然是完成了这次系统的设计。同时我们两人对汉明码的理解更深了一步。

在这次实验中,我们两人都积极对项目作出了贡献。在前期讨论的过程中,为了给自己更大的挑战,我们选择了难度比较大的汉明码编码,我们因此特别回顾了在“通信与网络”这门电子系必修课的内容,对汉明码有了一个更加全面的复习和认识,这对我们后来的设计工作带来了许多便利。

王冠儒同学主要负责编解码、调制模块的编写。他的verilog基本功非常扎实,代码风格也非常好,基本没有造成任何BUG。

万子牛同学主要负责解调模块编写和testbench的设计。

最后,感谢通信原理实验的老师和助教老师的付出,我们也希望我们能在今后的学习过程中,秉承着相同的严谨实验的精神,实现自身更多的价值。


\section{文件清单}

\begin{table}[h!]
  \caption{文件清单}\label{table:1}
  \begin{tabularx}{\textwidth}{>{\hsize=.6\hsize\raggedright\arraybackslash}X>{\raggedright\arraybackslash}X}\toprule
    \bfseries{文件} & \bfseries{描述}\\ \midrule
    \verb|encoder.v| & 编码器模块。 \\
    \verb|clk_code.v| & 原时钟信号的16分频时钟,用于生成编码模块的时钟输入。 \\ 
    \verb|FSK.v| & FSK调制模块。 \\
    \verb|deFSK.v| & 解调模块。 \\
    \verb|decode.v| & 解调过后的解码模块。 \\
    \verb|FSKSystem.v| & 顶层封装。 \\
    \verb|*_tb.v| & 对应模块的\verb|testbench|。 \\
    \verb|encode_noise.v| & 对encode进行了加噪处理。 \\
    \bottomrule
  \end{tabularx}
\end{table}



\end{document}
